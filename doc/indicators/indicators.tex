\documentclass[12pt,a4paper]{article}

% Package to include code
\usepackage{listings}
\usepackage{hyperref}
\usepackage{color}
\lstset{language=Python}
\lstset{numbers=none, basicstyle=\footnotesize,
  numberstyle=\tiny,keywordstyle=\color{blue},stringstyle=\ttfamily,showstringspaces=false}
\lstset{backgroundcolor=\color[rgb]{0.95 0.95 0.95}}
\lstdefinestyle{numbers}{numbers=left, stepnumber=1, numberstyle=\tiny, numbersep=10pt}
\lstdefinestyle{nonumbers}{numbers=none}


% Font selection: uncomment the next line to use the ``beton'' font
%\usepackage{beton}

% Font selection: uncomment the next line to use the ``times'' font
%\usepackage{times}

% Font for equations
\usepackage{euler}


%Package to define the headers and footers of the pages
\usepackage{fancyhdr}


%Package to include an index
\usepackage{index}

%Package to display boxes around texts. Used especially for the internal notes.
\usepackage{framed}

%PSTricks is a collection of PostScript-based TEX macros that is compatible
% with most TEX macro packages
\usepackage{pstricks}
\usepackage{pst-node}
\usepackage{pst-plot}
\usepackage{pst-tree}

%Package to display boxes around a minipage. Used especially to
%describe the biography of people.
\usepackage{boxedminipage}

%Package to include postscript figures
\usepackage{epsfig}

%Package for the bibliography
% \cite{XXX} produces Ben-Akiva et. al., 2010
% \citeasnoun{XXX} produces Ben-Akiva et al. (2010)
% \citeasnoun*{XXX} produces Ben-Akiva, Bierlaire, Bolduc and Walker (2010)
\usepackage[dcucite,abbr]{harvard}
\harvardparenthesis{none}\harvardyearparenthesis{round}

%Packages for advanced mathematics typesetting
\usepackage{amsmath,amsfonts,amssymb}

%Package to display trees easily
%\usepackage{xyling}

%Package to include smart references (on the next page, on the
%previous page, etc.) 
%%

%% Remove as it is not working when the book will be procesed by the
%% publisher.
%\usepackage{varioref}

%Package to display the euro sign
\usepackage[right,official]{eurosym}

%Rotate material, especially large table (defines sidewaystable)
\usepackage[figuresright]{rotating}

%Defines the subfigure environment, to obtain refs like Figure 1(a)
%and Figure 1(b). 
\usepackage{subfigure}

%Package for appendices. Allows subappendices, in particular
\usepackage{appendix}

%Package controling the fonts for the captions
\usepackage[font={small,sf}]{caption}

%Defines new types of columns for tabular ewnvironment
\usepackage{dcolumn}
\newcolumntype{d}{D{.}{.}{-1}}
\newcolumntype{P}[1]{>{#1\hspace{0pt}\arraybackslash}}
\newcolumntype{.}{D{.}{.}{9.3}}

%Allows multi-row cells in tables
\usepackage{multirow}

%Tables spaning more than one page
\usepackage{longtable}


%%
%%  Macros by Michel
%%

%Internal notes
\newcommand{\note}[1]{
\begin{framed}{}%
\textbf{\underline{Internal note}:} #1
\end{framed}}

%Use this version to turn off the notes
%\newcommand{\note}[1]{}


%Include a postscript figure . Note that the label is prefixed with
%``fig:''. Remember it when you refer to it.  
%Three arguments:
% #1 label
% #2 file (without extension)
% #3 Caption
\newcommand{\afigure}[3]{%
\begin{figure}[!tbp]%
\begin{center}%
\epsfig{figure=#2,width=0.8\textwidth}%
\end{center}
\caption{\label{fig:#1} #3}%
\end{figure}}






%Include two postscript figures side by side. 
% #1 label of the first figure
% #2 file for the first figure
% #3 Caption for the first figure
% #4 label of the second figure
% #5 file for the second figure
% #6 Caption for the first figure
% #7 Caption for the set of two figures
\newcommand{\twofigures}[7]{%
\begin{figure}[htb]%
\begin{center}%
\subfigure[\label{fig:#1}#3]{\epsfig{figure=#2,width=0.45\textwidth}}%
\hfill
\subfigure[\label{fig:#4}#6]{\epsfig{figure=#5,width=0.45\textwidth}}%
\end{center}
\caption{#7}%
\end{figure}}

%Include a figure generated by gnuplot using the epslatex output. Note that the label is prefixed with
%``fig:''. Remember it when you refer to it.  
 
%Three arguments:
% #1 label
% #2 file (without extension)
% #3 Caption
\newcommand{\agnuplotfigure}[3]{%
\begin{figure}[!tbp]%
\begin{center}%
\input{#2}%
\end{center}
\caption{\label{fig:#1} #3}%
\end{figure}}

%Three arguments:
% #1 label
% #2 file (without extension)
% #3 Caption
\newcommand{\asidewaysgnuplotfigure}[3]{%
\begin{sidewaysfigure}[!tbp]%
\begin{center}%
\input{#2}%
\end{center}
\caption{\label{fig:#1} #3}%
\end{sidewaysfigure}}


%Include two postscript figures side by side. 
% #1 label of the first figure
% #2 file for the first figure
% #3 Caption for the first figure
% #4 label of the second figure
% #5 file for the second figure
% #6 Caption for the second figure
% #7 Caption for the set of two figures
% #8 label for the whole figure
\newcommand{\twognuplotfigures}[7]{%
\begin{figure}[htb]%
\begin{center}%
\subfigure[\label{fig:#1}#3]{\input{#2}}%
\hfill
\subfigure[\label{fig:#4}#6]{\input{#5}}%
\end{center}
\caption{#7}%
\end{figure}}



%Include the description of somebody. Four arguments:
% #1 label
% #2 Name
% #3 file (without extension)
% #4 description
\newcommand{\people}[4]{
\begin{figure}[tbf]
\begin{boxedminipage}{\textwidth}
\parbox{0.40\textwidth}{\epsfig{figure=#3,width = 0.39\textwidth}}%\hfill
\parbox{0.59\textwidth}{%
#4% 
}%
\end{boxedminipage}
\caption{\label{fig:#1} #2}
\end{figure}
}

%Default command for a definition
% #1 label (prefix def:)
% #2 concept to be defined
% #3 definition
\newtheorem{definition}{Definition}
\newcommand{\mydef}[3]{%
\begin{definition}%
\index{#2|textbf}%
\label{def:#1}%
\textbf{#2} \slshape #3\end{definition}}

%Reference to a definitoin. Prefix 'def:' is assumed
\newcommand{\refdef}[1]{definition~\ref{def:#1}}


%Default command for a theorem, with proof
% #1: label (prefix thm:)
% #2: name of the theorem
% #3: statement
% #4: proof
\newtheorem{theorem}{Theorem}
\newcommand{\mytheorem}[4]{%
\begin{theorem}%
\index{#2|textbf}%
\index{Theorems!#2}%
\label{thm:#1}%
\textbf{#2} \sffamily \slshape #3
\end{theorem} \bpr #4 \epr \par}


%Default command for a theorem, without proof
% #1: label (prefix thm:)
% #2: name of the theorem
% #3: statement
\newcommand{\mytheoremsp}[3]{%
\begin{theorem}%
\index{#2|textbf}%
\index{Theorems!#2}%
\label{thm:#1}%
\textbf{#2} \sffamily \slshape #3
\end{theorem}}



%Put parentheses around the reference, as standard for equations
\newcommand{\req}[1]{(\ref{#1})}

%Short cut to make a column vector in math environment (centered)
\newcommand{\cvect}[1]{\left( \begin{array}{c} #1 \end{array} \right) }

%Short cut to make a column vector in math environment (right justified)
\newcommand{\rvect}[1]{\left( \begin{array}{r} #1 \end{array} \right) }

%A reference to a theorem. Prefix thm: is assumed for the label.
\newcommand{\refthm}[1]{theorem~\ref{thm:#1}}

%Reference to a figure. Prefix fig: is assumed for the label.
\newcommand{\reffig}[1]{Figure~\ref{fig:#1}}

%Smart reference to a figure. Prefix fig: is assumed for the label.
%\newcommand{\vreffig}[1]{Figure~\vref{fig:#1}}

%C in mathcal font for the choice set
\newcommand{\C}{\mathcal{C}}

%R in bold font for the set of real numbers
\newcommand{\R}{\mathbb{R}}

%N in bold font for the set of natural numbers
\newcommand{\N}{\mathbb{N}}

%C in mathcal font for the log likelihood
\renewcommand{\L}{\mathcal{L}}

%S in mathcal font for the subset S
\renewcommand{\S}{\mathcal{S}}

%To write an half in math envionment
\newcommand{\half}{\frac{1}{2}}

%Probability
\newcommand{\prob}{\operatorname{Pr}}

%Expectation
\newcommand{\expect}{\operatorname{E}}

%Variance
\newcommand{\var}{\operatorname{Var}}

%Covariance
\newcommand{\cov}{\operatorname{Cov}}

%Correlation
\newcommand{\corr}{\operatorname{Corr}}

%Span
\newcommand{\myspan}{\operatorname{span}}

%plim
\newcommand{\plim}{\operatorname{plim}}

%Displays n in bold (for the normal distribution?)
\newcommand{\n}{{\bf n}}

%Includes footnote in a table environment. Warning: the footmark is
%always 1.
\newcommand{\tablefootnote}[1]{\begin{flushright}
\rule{5cm}{1pt}\\
\protect\footnotemark[1]{\footnotesize #1}
\end{flushright}
}
\renewcommand*{\thefootnote}{\alph{footnote}}

%Defines the ``th'' as in ``19th'' to be a superscript
\renewcommand{\th}{\textsuperscript{th}}

%Begin and end of a proof
\newcommand{\bpr}{{\bf Proof.} \hspace{1 em}}
\newcommand{\epr}{$\Box$}


\title{Calculating  indicators with PythonBiogeme}
\author{Michel Bierlaire} 
\date{May 17, 2017}


\begin{document}


\begin{titlepage}
\pagestyle{empty}

\maketitle
\vspace{2cm}

\begin{center}
\small Report TRANSP-OR 170517\\ Transport and Mobility Laboratory \\ School of Architecture, Civil and Environmental Engineering \\ Ecole Polytechnique F\'ed\'erale de Lausanne \\ \verb+transp-or.epfl.ch+
\begin{center}
\textsc{Series on Biogeme}
\end{center}
\end{center}


\clearpage
\end{titlepage}

The package Biogeme (\texttt{biogeme.epfl.ch}) is designed to estimate
the parameters of various models using maximum likelihood estimation.
But it can also be used to extract indicators from an estimated
model. In this document, we describe how to calculate some indicators
particularly relevant in the context of discrete choice models: market
shares, revenues, elasticities, and willingness to pay. Clearly, the
use of the software is not restricted to these indicators, neither to
choice models. But these examples illustrate most of the
capabilities. 

\section{The model}

\begin{flushright}
See \href{http://biogeme.epfl.ch/examples/indicators/python/01nestedEstimation.py}{\lstinline$01nestedEstimation.py$} in Section~\ref{sec:01nestedEstimation}
\end{flushright}

We consider a case study involving a transportation mode choice model,
using revealed preference data collected in Switzerland in 2009 and
2010 (see \cite{AtaGlerBier2012_DISP}).
The model is a nested logit model with 3 alternatives: \emph{public
  transportation}, \emph{car} and \emph{slow modes}. The utility functions are defined as:
\begin{lstlisting}[style=nonumbers,backgroundcolor=]
V_PT = BETA_TIME_FULLTIME * TimePT_scaled * fulltime +
       BETA_TIME_OTHER * TimePT_scaled * notfulltime +
       BETA_COST * MarginalCostPT_scaled
V_CAR = ASC_CAR +
        BETA_TIME_FULLTIME * TimeCar_scaled * fulltime +
        BETA_TIME_OTHER * TimeCar_scaled * notfulltime +
        BETA_COST * CostCarCHF_scaled
V_SM = ASC_SM +
       BETA_DIST_MALE * distance_km_scaled * male  +
       BETA_DIST_FEMALE * distance_km_scaled * female +
       BETA_DIST_UNREPORTED * distance_km_scaled * unreportedGender
\end{lstlisting}
where
\lstinline@ASC_CAR@, 
\lstinline@ASC_SM@, 
\lstinline@BETA_TIME_FULLTIME@, 
\lstinline@BETA_TIME_OTHER@, 
\lstinline@BETA_DIST_MALE@, 
\lstinline@BETA_DIST_FEMALE@, 
\lstinline@BETA_DIST_UNREPORTED@, 
\lstinline@BETA_COST@, 
are parameters to be estimated,  
\lstinline@TimePT_scale@,  
\lstinline@MarginalCostPT_scaled@,  
\lstinline@TimeCar_scale@, 
\lstinline@CostCarCHF_scale@, 
\lstinline@distance_km_scale@
are attributes and
\lstinline@fulltime@, 
\lstinline@notfulltime@,  
\lstinline@male@, 
\lstinline@female@, 
\lstinline@unreportedGender@ are socio-economic characteristics.
The two alternatives ``public transportation'' and ``slow modes'' are
grouped into a nest. 
The complete specification is available in the file \href{http://biogeme.epfl.ch/examples/indicators/python/01nestedEstimation.py}{\lstinline$01nestedEstimation.py$}, reported in
Section~\ref{sec:01nestedEstimation}. We refer the reader to
\citeasnoun{Bier16a} for an introduction to the syntax. 

The parameters are estimated using PythonBiogeme. Their values are
reported in Table~\ref{tab:estimatedParameters}. A file named
\lstinline$01nestedEstimation_param.py$ is also generated. It contains
the values of the estimated parameters written in PythonBiogeme
syntax, as well as the code necessary to perform a sensitivity
analysis. This code provides the variance-covariance matrix of the
estimates.

\begin{table}
  \begin{tabular}{l}
\begin{tabular}{rlr@{.}lr@{.}lr@{.}lr@{.}l}
         &                       &   \multicolumn{2}{l}{}    & \multicolumn{2}{l}{Robust}  &     \multicolumn{4}{l}{}   \\
Parameter &                       &   \multicolumn{2}{l}{Coeff.}      & \multicolumn{2}{l}{Asympt.}  &     \multicolumn{4}{l}{}   \\
number &  Description                     &   \multicolumn{2}{l}{estimate}      & \multicolumn{2}{l}{std. error}  &   \multicolumn{2}{l}{$t$-stat}  &   \multicolumn{2}{l}{$p$-value}   \\

\hline

1 & \lstinline@ASC_CAR@ & 0&261 & 0&100 & 2&61 & 0&01\\
2 & \lstinline@ASC_SM@ & 0&0590 & 0&217 & 0&27 & 0&79\\
3 & \lstinline@BETA_COST@ & -0&716 & 0&138 & -5&18 & 0&00\\
4 & \lstinline@BETA_DIST_FEMALE@ & -0&831 & 0&193 & -4&31 & 0&00\\
5 & \lstinline@BETA_DIST_MALE@ & -0&686 & 0&161 & -4&27 & 0&00\\
6 & \lstinline@BETA_DIST_UNREPORTED@ & -0&703 & 0&196 & -3&58 & 0&00\\
7 & \lstinline@BETA_TIME_FULLTIME@ & -1&60 & 0&333 & -4&80 & 0&00\\
8 & \lstinline@BETA_TIME_OTHER@ & -0&577 & 0&296 & -1&95 & 0&05\\
9 & \lstinline@NEST_NOCAR@ & 1&53 & 0&306 & 1&73\footnotemark[1] & 0&08\\

\hline
\end{tabular}
\\
\begin{tabular}{rcl}
\multicolumn{3}{l}{\bf Summary statistics}\\
\multicolumn{3}{l}{ Number of observations = $1906$} \\
\multicolumn{3}{l}{ Number of excluded observations = $359$} \\
\multicolumn{3}{l}{ Number of estimated  parameters = $9$} \\
 $\mathcal{L}(\beta_0)$ &=&  $-2093.955$ \\
 $\mathcal{L}(\hat{\beta})$ &=& $-1298.498 $  \\
 $-2[\mathcal{L}(\beta_0) -\mathcal{L}(\hat{\beta})]$ &=& $1590.913$ \\
    $\rho^2$ &=&   $0.380$ \\
    $\bar{\rho}^2$ &=&    $0.376$ \\
\end{tabular}
  \end{tabular}
\tablefootnote{$t$-test against 1} 
\caption{\protect\label{tab:estimatedParameters}Nested logit model: estimated parameters}
\end{table}

\section{Market shares and revenues}

\begin{flushright}
See \href{http://biogeme.epfl.ch/examples/indicators/python/02nestedSimulation.py}{\lstinline$02nestedSimulation.py$} in Section~\ref{sec:02nestedSimulation}
\end{flushright}


Once the model has been estimated, it must be used to derive useful
indicators. PythonBiogeme provides a simulation feature for this
purpose. We start by describing how to calculate market shares using
sample enumeration. It is necessary to have a sample of individuals
from the population. For each of them, the value of each of the
variables involved in the model must be known.  Note that it is possible to use the same sample
that what used for estimation, but only if it contains revealed
preferences data. Indeed, the calculation of indicators require real
values for the variables, not values that have been engineered to the
sake of estimating parameters, like in stated preferences data. It is the procedure used in this document. 


More formally, consider a choice model $P_n(i|x_n, \C_n)$ providing
the probability that individual $n$ chooses alternative $i$ within the
choice set $\C_n$, given the explanatory variables $x_n$.  In order to
calculate the market shares in the population of size $N$, a sample of
$N_s$ individuals is drawn. As it is rarely possible to draw from the
population with equal sampling probability, it is assumed that
stratified sampling has been used, and that each individual $n$ in the
sample is associated with a weight $w_n$ correcting for sampling
biases. The weights are normalized such that
\begin{equation}
  \label{eq:normalizingWeights}
N_s = \sum_{n=1}^{N_s} w_n.
\end{equation}
An estimator of the market share of alternative $i$ in the population is
\begin{equation}
  \label{eq:marketShare}
W_i = \frac{1}{N_s} \sum_{n=1}^{N_s} w_n P_n(i|x_n, \C_n).
\end{equation}
If the alternative $i$ involves a price variable $p_{in}$, the expected revenue
generated by $i$ is
\begin{equation}
\label{eq:revenues}
R_i = \frac{N}{N_s} \sum_{n=1}^{N_s} w_n p_{in} P_n(i|x_n, p_{in}, \C_n).
\end{equation}
In practice, the size of the population is rarely known, and the above
quantity is used only in the context of price optimization. In this
case, the factor $N/N_s$ can be omitted. 

To calculate \req{eq:marketShare} and \req{eq:revenues} with
PythonBiogeme, a specification file must be prepared. In our example,
the file \href{http://biogeme.epfl.ch/examples/indicators/python/02nestedSimulation.py}{\lstinline$02nestedSimulation.py$},
reported in Section~\ref{sec:02nestedSimulation}, has been produced
as follows:
\begin{enumerate}
\item Start with a copy of the model estimation file \href{http://biogeme.epfl.ch/examples/indicators/python/01nestedEstimation.py}{\lstinline$01nestedEstimation.py$}.
\item Replace all \lstinline$Beta$ statements by the equivalent
  statements including the estimated values in the file \lstinline$01nestedEstimation_param.py$.
\item Copy and paste the code for the sensitivity analysis, that is
  \begin{itemize}
  \item the names of the parameters: the line starting with \lstinline$names=...$
  \item the values of the variance-covariance matrix:
    the line starting with \lstinline$values=...$
  \item the definition of the matrix itself:
\begin{lstlisting}[style=nonumbers,backgroundcolor=]
vc = bioMatrix(9,names,values)
BIOGEME OBJECT.VARCOVAR = vc
\end{lstlisting}
  \end{itemize}
\item Remove the statement related to the estimation:
\begin{lstlisting}
  BIOGEME_OBJECT.ESTIMATE = Sum(logprob,'obsIter')
\end{lstlisting}
\item Replace it by the statement for simulation:
\begin{lstlisting}
  BIOGEME_OBJECT.SIMULATE = Enumerate(simulate,'obsIter')
\end{lstlisting}
The \lstinline$simulate$ variable must be a dictionary describing what
has to be calculated during the sample enumeration. In this case, we
calculate, for each individual in the sample, the choice probability
of each alternative. We also calculate the expected revenue generated
by each individual for the public transportation companies, using the following statement:
\begin{lstlisting}
simulate = {'Prob. car': prob_car,
            'Prob. public transportation': prob_pt,
            'Prob. slow modes':prob_sm,
            'Revenue public transportation':
                   prob_pt * MarginalCostPT}
\end{lstlisting}
Each entry of this dictionary corresponds to a quantity that will be
calculated. The key of the entry is a string, that will be used for
the reporting. The value must be a valid formula describing the
calculation. In our example, we have defined
\begin{lstlisting}
prob_pt = nested(V,av,nests,0)
prob_car = nested(V,av,nests,1)
prob_sm = nested(V,av,nests,2)
\end{lstlisting}
calculating the choice probability of each alternative as provided by
the nested logit model.
  \end{enumerate}

In the output of the estimation (see the file
\href{http://biogeme.epfl.ch/examples/indicators/python/01nestedEstimation.html}{\lstinline$01nestedEstimation.html$}), the sum of all weights have been
calculated using the statement
\begin{lstlisting}
  BIOGEME_OBJECT.STATISTICS['Sum of weights'] = Sum(Weight,'obsIter')
\end{lstlisting}
The reported result is 0.814484. Therefore, in order to verify
\req{eq:normalizingWeights}, we introduce the following statements:
\begin{lstlisting}
theWeight = Weight * 1906 / 0.814484
BIOGEME_OBJECT.WEIGHT = theWeight
\end{lstlisting}
as there are 1906 entries in the data file.

The following statements are included for the calculation of
elasticities and will be used later (see Section~\ref{sec:elasticities} for more details):
\begin{lstlisting}
  BIOGEME_OBJECT.STATISTICS['Normalization for elasticities PT'] = 
    Sum(theWeight * prob_pt ,'obsIter')
  BIOGEME_OBJECT.STATISTICS['Normalization for elasticities CAR'] =
    Sum(theWeight * prob_car ,'obsIter')
  BIOGEME_OBJECT.STATISTICS['Normalization for elasticities SM'] =
    Sum(theWeight * prob_sm ,'obsIter')
\end{lstlisting}

The simulation is performed using the statement
\begin{lstlisting}
  pythonbiogeme 02nestedSimulation optima.dat
\end{lstlisting}
It generates the file \href{http://biogeme.epfl.ch/examples/indicators/python/02nestedSimulation.html}{\lstinline$02nestedSimulation.html$}, that
contains the following sections:
\begin{itemize}
  \item The preamble reports information about the version of
    PythonBiogeme, useful URLs and the names of the files involved in
    the run.
  \item Statistics: this section is the same as for the estimation,
    and reports the requested statistics:

\begin{lstlisting}    
                             Alt. 0 available:	1906
                                Alt. 0 chosen:	536
                             Alt. 1 available:	1906
                                Alt. 1 chosen:	1256
                             Alt. 2 available:	1906
                                Alt. 2 chosen:	114
Cte loglikelihood (only for full choice sets):	-1524.92
                              Gender: females:	871
                                Gender: males:	943
                           Gender: unreported:	92
           Normalization for elasticities CAR:	1244.77
            Normalization for elasticities PT:	535.086
            Normalization for elasticities SM:	126.147
                           Null loglikelihood:	-2093.96
                            Number of entries:	1906
                        Occupation: full time:	798
                               Sum of weights:	0.814484
\end{lstlisting}
\item The simulation report contains two parts: the aggregate values,
  and the detailed records. We start by describing the latter.
It reports, for each row of the sample file, the weight $w_n$ (last column) and, for each entry in
the dictionary defined by \lstinline$BIOGEME_OBJECT.SIMULATE$
\begin{enumerate}
\item the calculated quantity,
\item the 90\% confidence interval for this quantity. It is calculated
  using simulation. As the estimates have been obtained from
  maximum likelihood, they are (asymptotically) normally
  distributed. Therefore, we draw from a multivariate normal
  distribution $N(\widehat{\beta},\widehat{\Sigma})$, where
  $\widehat{\beta}$ is the vector of estimated parameters, and
  $\widehat{\Sigma}$ is the variance-covariance matrix defined by the
  \lstinline$BIOGEME OBJECT.VARCOVAR$ statement. The number of draws
  is controlled by the parameter
  \lstinline$NbrOfDrawsForSensitivityAnalysis$. The requested quantity
  is calculated for each realization, and the 5\% and the 95\%
  quantiles of the obtained simulated values are reported to generate
  the 90\% confidence interval.
  Note that the confidence interval is reported only if the statement
  \begin{lstlisting}
    BIOGEME_OBJECT.VARCOVAR = vc
  \end{lstlisting}
  is present. If you do not need the confidence intervals, simply
  remove this statement from the \lstinline$.py$ file.
\end{enumerate}
  \item Simulation report: aggregate values. For each calculated
    quantity, aggregate indicators are calculated. Denote by $z_n$ the
    calculated quantity (in this case, the probability that individual $n$ chooses
    the car alternative, for instance). Then, the following aggregate
    values are reported, together with the associated confidence
    interval (if requested):
    \begin{itemize}
    \item Total:
      \begin{equation}
\sum_{n=1}^{N_s} z_n.
      \end{equation}
    \item Weighted total:
      \begin{equation}
\sum_{n=1}^{N_s} w_n z_n.
      \end{equation}
    \item Average:
      \begin{equation}
\frac{1}{N_s}\sum_{n=1}^{N_s} z_n.
      \end{equation}
    \item Weighted average:
      \begin{equation}
\frac{1}{N_s}\sum_{n=1}^{N_s} w_n z_n.
      \end{equation}
    \item Non zeros:
      \begin{equation}
        \sum_{n=1}^{N_s} \delta(z_n \neq 0),
      \end{equation}
      where
      \begin{equation}
        \delta(z_n \neq 0) = \left\{
        \begin{array}{ll}
          1 & \text{if } z_n \neq 0, \\
          0 & \text{otherwise}.
        \end{array}
        \right.
      \end{equation}
    \item Non zeros average:
      \begin{equation}
\frac{\sum_{n=1}^{N_s} z_n}{ \sum_{n=1}^{N_s} \delta(z_n \neq 0)}.
      \end{equation}
    \item Weighted non zeros average:
      \begin{equation}
\frac{\sum_{n=1}^{N_s} w_n z_n }{ \sum_{n=1}^{N_s} \delta(z_n \neq 0)}.
      \end{equation}
    \item Minimum:
      \begin{equation}
\min_n z_n.
      \end{equation}
    \item Maximum:
      \begin{equation}
\max_n z_n.
      \end{equation}
    \end{itemize}
\end{itemize}

Therefore, the result of \req{eq:marketShare} is available in the
row ``Weighted average''. In this example, the market shares are:
\begin{itemize}
\item car: 65.3078\% (confidence interval: [60.5884\%,69.0407\%]),
  \item public transportation: 28.0738\% (confidence interval:
    [23.603\%,32.391\%],
    \item slow modes: 6.61844\% (confidence interval:
      4.54637\%,10.417\%).
\end{itemize}
The result of \req{eq:revenues} is obtained in the row ``Weighted
total''. In this case, the expected revenue (generated by the
individuals in the sample) is 3018.29 (confidence interval: [2442.87,3826.36]).

\section{Elasticities}
\label{sec:elasticities}
Consider now one of the variables involved in the model, for instance
$x_{ink}$, the $k$th variable associated by individual $n$ to
alternative $i$. The
objective is to anticipate the impact of a change of the value of this
variable on the choice of individual $n$,  and subsequently on the market share of
alternative $i$.

\subsection{Point elasticities}

If the variable is continuous, we assume that the relative (infinitesimal) change of
the variable is the same for every individual in the population,  that
is
\begin{equation}
  \label{eq:uniformChange}
\frac{\partial x_{ink}}{x_{ink}} = \frac{\partial x_{ipk}}{x_{ipk}} = 
\frac{\partial x_{ik}}{x_{ik}}, 
\end{equation}
where
\begin{equation}
  \label{eq:avgx}
x_{ik} = \frac{1}{N_s} \sum_{n=1}^{N_s}{x_{ink}}.
\end{equation}
The \emph{disaggregate direct point elasticity} of the model with respect to
the variable $x_{ink}$ is defined as
\begin{equation}
\label{eq:disagElasticity}
  E_{x_{ink}}^{P_n(i)} = \frac{\partial P_n(i|x_n, \C_n)}{\partial
  x_{ink}} \frac{x_{ink}}{P_n(i|x_n, \C_n)}.
\end{equation}
It is called
\begin{itemize}
\item disaggregate,  because it refers to the choice model related to a
  specific individual, 
\item direct,  because it measures the impact of a change of an
    attribute of alternative $i$ on the choice probability of the
    same alternative, 
\item point,  because we consider an infinitesimal change of the
  variable. 
\end{itemize}
The \emph{aggregate direct point elasticity} of the model with
respect to the average value $x_{ik}$ is defined as
\begin{equation}
E_{x_{ik}}^{W_i} = \frac{\partial W_i}{\partial x_{ik}} \frac{x_{ik}}{W_i}.
\end{equation}
Using \req{eq:marketShare},  we obtain
\begin{equation}
E_{x_{ik}}^{W_i} = \frac{1}{N_s}  \sum_{n=1}^{N_s} w_n \frac{\partial
  P_n(i|x_n, \C_n)}{\partial x_{ik}} \frac{x_{ik}}{W_i}.
\end{equation}
From \req{eq:uniformChange},  we obtain
\begin{equation}
E_{x_{ik}}^{W_i} = \frac{1}{N_s}  \sum_{n=1}^{N_s} w_n \frac{\partial
  P_n(i|x_n, \C_n)}{\partial x_{ink}} \frac{x_{ink}}{W_i} =
\frac{1}{N_s}  \sum_{n=1}^{N_s} w_n E_{x_{ink}}^{P_n(i)}  \frac{P_n(i|x_n, \C_n)}{W_i}, 
\end{equation}
where the second equation is derived from \req{eq:disagElasticity}.
Using \req{eq:marketShare} again,  we obtain
\begin{equation}
\label{eq:aggDisagg}
  E_{x_{ik}}^{W_i} =  \sum_{n=1}^{N_s} 
E_{x_{ink}}^{P_n(i)}  \frac{w_n P_n(i|x_n, \C_n)}{ \sum_{n=1}^{N_s} w_n P_n(i|x_n, \C_n)}.
\end{equation}
This equation shows that the calculation of aggregate elasticities
involves a weighted sum of disaggregate elasticities. However,  the
weight is not $w_n$ as for the market share,  but a normalized version
of $w_n P_n(i|x_n, \C_n)$.

The \emph{disaggregate cross point elasticity} of the model with respect to
the variable $x_{jnk}$ is defined as
\begin{equation}
\label{eq:disagCrossElasticity}
  E_{x_{jnk}}^{P_n(i)} = \frac{\partial P_n(i|x_n, \C_n)}{\partial
  x_{jnk}} \frac{x_{jnk}}{P_n(i|x_n, \C_n)}.
\end{equation}
It is called \emph{cross} elasticity because it measures the sensitivity
of the model for alternative $i$ with respect to a  modification of
the attribute of another alternative.



\subsection{Arc elasticities}

A similar derivation can be done for arc elasticities. In this case, 
the relative change of the variable is not infinitesimal anymore. The
idea is to analyze a before/after scenario. The variable $x_{ink}$ in
the before scenario becomes $x_{ink} + \Delta x_{ink}$ in the after scenario.
As above,  we assume that the relative change of
the variable is the same for every individual in the population,  that
is
\begin{equation}
  \label{eq:uniformChangeArc}
\frac{\Delta x_{ink}}{x_{ink}} = \frac{\Delta x_{ipk}}{x_{ipk}} = 
\frac{\Delta x_{ik}}{x_{ik}}, 
\end{equation}
where $x_{ik}$ is defined by \req{eq:avgx}.
The \emph{disaggregate direct arc elasticity} of the model with respect to
the variable $x_{ink}$ is defined as
\begin{equation}
\label{eq:disagElasticityArc}
  E_{x_{ink}}^{P_n(i)} = \frac{\Delta P_n(i|x_n, \C_n)}{\Delta
  x_{ink}} \frac{x_{ink}}{P_n(i|x_n, \C_n)}.
\end{equation}
The \emph{aggregate direct arc elasticity} of the model with
respect to the average value $x_{ik}$ is defined as
\begin{equation}
E_{x_{ik}}^{W_i} = \frac{\Delta W_i}{\Delta x_{ik}} \frac{x_{ik}}{W_i}.
\end{equation}
The two quantities are also related by \req{eq:aggDisagg},  following
the exact same derivation as for the point elasticity.

\subsection{Using PythonBiogeme for point elasticities}

\begin{flushright}
See \href{http://biogeme.epfl.ch/examples/indicators/python/03nestedElasticities.py}{\lstinline$03nestedElasticities.py$} in Section~\ref{sec:03nestedElasticities}
\end{flushright}

The calculation of \req{eq:disagElasticity} involves derivatives. For
simple models such as logit, the analytical formula of these
derivatives can easily be derived. However, their derivation for
advanced models can be tedious. It is common to make mistakes in the
derivation itself, and even more common to make mistakes in the
implementation. Therefore, PythonBiogeme provides an operator that
calculates the derivative of a formula. It is illustrated in the
file \href{http://biogeme.epfl.ch/examples/indicators/python/03nestedElasticities.py}{\lstinline$03nestedElasticities.py$}, reported in
Section~\ref{sec:03nestedElasticities}. The statements that trigger
the calculation of the elasticities are:
\begin{lstlisting}
elas_pt_time = Derive(prob_pt,'TimePT') * TimePT / prob_pt
elas_pt_cost = Derive(prob_pt,'MarginalCostPT') * MarginalCostPT / prob_pt
elas_car_time = Derive(prob_car,'TimeCar') * TimeCar / prob_car
elas_car_cost = Derive(prob_car,'CostCarCHF') * CostCarCHF / prob_car
elas_sm_dist = Derive(prob_sm,'distance_km') * distance_km / prob_sm
\end{lstlisting}
The above syntax should be self-explanatory. But there is an important
aspect to take into account. In the context of the estimation of the
parameters of the model, the variables have been scaled in order to
improve the numerical properties of the likelihood function, using
statements like
\begin{lstlisting}
TimePT_scaled = DefineVariable('TimePT_scaled', TimePT / 200 )
\end{lstlisting}
The \lstinline$DefineVariable$ operator is designed to preprocess the
data file, and can be seen as a way to add another column in the data
file, defining a new variable. However, the relationship between  the
new variable and the original one is lost. Therefore,
PythonBiogeme is not able to properly calculate the
derivatives. In this example, the variable of interest is
\lstinline$TimePT$, not \lstinline$TimePT_scaled$. And their
relationship must be explicitly known to correctly calculate the derivatives.   Consequently, all statements such as 
\begin{lstlisting}
TimePT_scaled = DefineVariable('TimePT_scaled', TimePT / 200 )
\end{lstlisting}
should be replaced by statements such as
\begin{lstlisting}
TimePT_scaled  = TimePT   /  200 
\end{lstlisting}
in order to maintain the analytical structure of the formula to be derived.


The aggregate point elasticities can be obtained by aggregating the
disaggregate elasticities, using \req{eq:aggDisagg}. This requires the
calculation of the normalization factors
\begin{equation}
\sum_{n=1}^{N_s} w_n P_n(i|x_n, \C_n).
\end{equation}
This has been performed during the previous simulation using the
statements
\begin{lstlisting}
BIOGEME_OBJECT.STATISTICS['Normalization for elasticities PT'] = \
                    Sum(theWeight * prob_pt ,'obsIter')
BIOGEME_OBJECT.STATISTICS['Normalization for elasticities CAR'] = \
                    Sum(theWeight * prob_car ,'obsIter')
BIOGEME_OBJECT.STATISTICS['Normalization for elasticities SM'] = \
                    Sum(theWeight * prob_sm ,'obsIter')
\end{lstlisting}
Therefore, we have now included the following statements:
\begin{lstlisting}
normalization_pt  = 535.086
normalization_car = 1244.77
normalization_sm = 126.147
\end{lstlisting}
The quantities that must be calculated for each individual
in order to derive the aggregate elasticities, correspond to the
following entries in the dictionary:
\begin{lstlisting}
'Agg. Elast. PT - Time': elas_pt_time * prob_pt / normalization_pt,
'Agg. Elast. PT - Cost': elas_pt_cost * prob_pt / normalization_pt,
'Agg. Elast. Car - Time': elas_car_time * prob_car / normalization_car,
'Agg. Elast. Car - Cost': elas_car_cost * prob_car / normalization_car,
'Agg. Elast. Slow modes - Distance': elas_sm_dist * prob_sm / normalization_sm
\end{lstlisting}
Note that the weights have not been included in the above formula, so
that 
the values of the aggregate elasticities can be found in the row ``Weighted total'':
\begin{itemize}
\item Car --- cost: -0.0906321,
\item Car --- travel time: -0.0440771,
\item Public transportation --- cost: -0.320246,
\item Public transportation --- travel time: -0.274315,
\item Slow modes --- distance: -1.09095.
\end{itemize}
Equivalently, we could have used statements like
\begin{lstlisting}
'Agg. Elast. PT - Time': theWeight * elas_pt_time * prob_pt / normalization_pt,
\end{lstlisting}
and the aggregate value would have been found in the row ``Total''
instead of ``Weighted total'.
Note also that we have omitted to report the confidence intervals in this
example, by commenting out the statement:
\begin{lstlisting}
#BIOGEME_OBJECT.VARCOVAR = vc
\end{lstlisting}

The results are found in the file \href{http://biogeme.epfl.ch/examples/indicators/python/03nestedElasticities.html}{\lstinline$03nestedElasticities.html$}.

\subsection{Using PythonBiogeme for cross elasticities}

\begin{flushright}
See \href{http://biogeme.epfl.ch/examples/indicators/python/04nestedElasticities.py}{\lstinline$04nestedElasticities.py$} in Section~\ref{sec:04nestedElasticities}
\end{flushright}



The calculation of \req{eq:disagCrossElasticity} is performed in a
similar way as the direct elasticities \req{eq:disagElasticity}, using the following statements:
\begin{lstlisting}
elas_car_cost = Derive(prob_car,'MarginalCostPT') * MarginalCostPT / prob_car
elas_car_time = Derive(prob_car,'TimePT') * TimePT / prob_car
elas_pt_cost = Derive(prob_pt,'CostCarCHF') * CostCarCHF / prob_pt
elas_pt_time = Derive(prob_pt,'TimeCar') * TimeCar / prob_pt
\end{lstlisting}
They calculate the following elasticities:
\begin{itemize}
\item choice of car with respect to the marginal cost of public
  transportation,
\item choice of car with respect to travel time by public
  transportation,
\item choice of public transportation with respect to cost of the
    car,
\item choice of public transportation with respect to travel time by
  car.
\end{itemize}
The corresponding aggregate elasticities are calculated exactly like
for the direct case, and their values can be found in the row
``Weighted total''.
\begin{itemize}
\item Agg. Elast. Car - Cost PT: 0.123008
\item Agg. Elast. Car - Time PT: 0.106567
\item Agg. Elast. PT - Cost car: 0.199984
\item Agg. Elast. PT - Time car: 0.0953097
\end{itemize}
Note that these values are now positive. Indeed, when the travel time
or travel cost of a competing mode increase, the market share
increases.

The results are found in the file \href{http://biogeme.epfl.ch/examples/indicators/python/04nestedElasticities.html}{\lstinline$04nestedElasticities.html$}.

\subsection{Using PythonBiogeme for arc elasticities}

\begin{flushright}
See \href{http://biogeme.epfl.ch/examples/indicators/python/05nestedElasticities.py}{\lstinline$05nestedElasticities.py$} in Section~\ref{sec:05nestedElasticities}
\end{flushright}


Arc elasticities require a before and after scenarios. In this case,
we calculate the sensitivity of the market share of the slow modes
alternative when there is a uniform increase of 1 kilometer.

The ``before'' scenario is represented by the same model as above. The
after scenario is modeled using the following statements:
\begin{lstlisting}
delta_dist = 1
distance_km_scaled_after  = (distance_km + delta_dist)   /  5
V_SM_after = ASC_SM + \
       BETA_DIST_MALE * distance_km_scaled_after * male + \
       BETA_DIST_FEMALE * distance_km_scaled_after * female + \
       BETA_DIST_UNREPORTED * distance_km_scaled_after * unreportedGender
V_after = {0: V_PT,
           1: V_CAR,
           2: V_SM_after}
prob_sm_after = nested(V_after,av,nests,2)
\end{lstlisting}
Then, the arc elasticity is calculated as
\begin{lstlisting}
elas_sm_dist = \
 (prob_sm_after - prob_sm) * distance_km / (prob_sm * delta_dist) 
\end{lstlisting}
The aggregate elasticity is calculated as explained above. It is equal
here to
-1.00708, and the confidence interval is [-1.7212,-0.562574].

The results are found in the file \href{http://biogeme.epfl.ch/examples/indicators/python/05nestedElasticities.html}{\lstinline$05nestedElasticities.html$}.

\section{Willingness to pay}

\begin{flushright}
See \href{http://biogeme.epfl.ch/examples/indicators/python/06nestedWTP.py}{\lstinline$06nestedWTP.py$} in Section~\ref{sec:06nestedWTP}
\end{flushright}


If the model contains a cost or price variable (like in this example),
it is possible to analyze the trade-off between any variable and money. 
This reflects the willingness of the decision maker to pay for a modification of another variable of the model.
A typical example in transportation is the \emph{value of time}, that
is the amount of money a traveler is willing to pay in order to
decrease her travel time.

Let $c_{in}$ be the cost of alternative $i$ for individual $n$.
Let $x_{ink}$ be the value of another variable of the model. 
Let $V_{in}(c_{in},x_{ink})$ be the value of the utility function. 
Consider a scenario where the variable of interest takes the value
$x_{ink} + \delta^x_{ink}$. 
We denote by $\delta^c_{in}$ the additional  cost  that would achieve the same utility, that is
\begin{equation}
  \label{eq:wtpEquation}
V_{in}(c_{in}+\delta^c_{in},x_{ink}+\delta^x_{ink}) = V_{in}(c_{in},x_{ink}).
\end{equation}
The willingness to pay to increase the value of $x_{ink}$ is defined
as the additional cost per unit of $x$, that is 
\begin{equation}
  \label{eq:wtpDiscrete}
  \delta^c_{in}/\delta^x_{ink},
\end{equation}
and is obtained by solving Equation~\req{eq:wtpEquation}.
If $x_{ink}$ and $c_{in}$ appear linearly in the utility function, that
is if
\begin{equation}
V_{in}(c_{in},x_{ink}) = \beta_c c_{in} + \beta_x x_{ink} + \cdots,
\end{equation}
and
\begin{equation}
V_{in}(c_{in}+\delta^c_{in},x_{ink}+\delta^x_{ink}) = \beta_c (c_{in}+\delta^c_{in}) + \beta_x (x_{ink}+\delta^x_{ink}) + \cdots.
\end{equation}
Therefore, \req{eq:wtpDiscrete} is
\begin{equation}
  \label{eq:wtpLinear}
  \delta^c_{in}/\delta^x_{ink} = -\beta_x / \beta_c.
\end{equation}
If $x_{ink}$ is a continuous variable, and if $V_{in}$ is
differentiable in $x_{ink}$ and $c_{in}$, we can invoke  Taylor's
theorem in \req{eq:wtpEquation}:
\begin{equation}
\begin{aligned}
V_{in}(c_{in},x_{ink})&= V_{in}(c_{in}+\delta^c_{in},x_{ink}+\delta^x_{ink})\\ &\approx V_{in}(c_{in},x_{ink}) + \delta^c_{in} \frac{\partial V_{in}}{\partial c_{in}}(c_{in},x_{ink})+ \delta^x_{ink} \frac{\partial V_{in}}{\partial x_{ink}}(c_{in},x_{ink})
\end{aligned}
\end{equation}
Therefore, the willingness to pay is equal to 
\begin{equation}
  \label{eq:wtpContinuous}
\frac{\delta^c_{in}}{ \delta^x_{ink}} = - \frac{(\partial V_{in}/\partial x_{ink})(c_{in},x_{ink})}{(\partial V_{in}/\partial c_{in})(c_{in},x_{ink})}.
\end{equation}
Note that if $x_{ink}$ and $c_{in}$ appear linearly in the utility
function, \req{eq:wtpContinuous} is the same as \req{eq:wtpLinear}.
If we consider now a scenario where the variable under interest takes the value
$x_{ink} - \delta^x_{ink}$, the same derivation leads to the
willingness to pay to \emph{decrease} the value of $x_{ink}$:
\begin{equation}
  \label{eq:wtpContinuousDecrease}
\frac{\delta^c_{in}}{ \delta^x_{ink}} = \frac{(\partial V_{in}/\partial x_{ink})(c_{in},x_{ink})}{(\partial V_{in}/\partial c_{in})(c_{in},x_{ink})}.
\end{equation}
The calculation of the value of time corresponds to such a scenario:
\begin{equation}
\frac{\delta^c_{in}}{ \delta^t_{in}} =  \frac{(\partial V_{in}/\partial t_{in})(c_{in},t_{in})}{(\partial V_{in}/\partial c_{in})(c_{in},t_{in})} = \frac{\beta_t}{\beta_c},
\end{equation}
where the last equation assumes that $V$ is linear in these variables.
Note that, in this special case of linear utility functions, the value
of time is constant across individuals, and is also independent of
$\delta^t_{in}$. This is not true in general.

The calculation of \req{eq:wtpContinuousDecrease} involves the
calculation of derivatives. It is done in Pythonbiogeme using the
following statements:
\begin{lstlisting}
WTP_PT_TIME = Derive(V_PT,'TimePT') / Derive(V_PT,'MarginalCostPT')
WTP_CAR_TIME = Derive(V_CAR,'TimeCar') / Derive(V_CAR,'CostCarCHF')
\end{lstlisting}
The full specification file can be found in
Section~\ref{sec:06nestedWTP}. The aggregate values are found in the
``Weighted average'' row of the report file:
3.95822 CHF/hour (confidence interval:
[1.98696,7.81565]). Note that this value is abnormally low, which is a
sign of a potential poor specification of the model. 
Note also that, with this specification, the value of time is the same for
car and public transportation, as the coefficients of the time and
cost variables are generic.

Finally, it is important to look at the distribution of the
willingness to pay in the population/sample. The detailed records of
the report file allows  to do so. It is easy to drag and drop the
HTML report file into your favorite spreadsheet software in order to
perform additional statistics. In this example, the value of time
takes two values, depending on the employment status of the individual:
\begin{itemize}
\item Full time: 6.68992 (confidence interval:	[4.15056, 11.1866])
\item Not full time: 2.41847 (confidence interval: [0.829511, 5.91561])
\end{itemize}

The results are found in the file \href{http://biogeme.epfl.ch/examples/indicators/python/06nestedWTP.html}{\lstinline$06nestedWTP.html$}.

\section{Conclusion}

PythonBiogeme is a flexible tool that allows to extract useful
indicators from complex models. In this document, we have presented
how some indicators relevant for discrete choice models  can be
generated. The HTML format of the report  allows to display
the report in your favorite browser. It also allows to import the
generated values in a spreadsheet for more manipulations. 

\clearpage

\appendix

\section{Complete specification files}

\subsection{\lstinline$01nestedEstimation.py$}
\label{sec:01nestedEstimation}

Available at \href{http://biogeme.epfl.ch/examples/indicators/python/01nestedEstimation.py}{\lstinline$biogeme.epfl.ch/examples/indicators/python/01nestedEstimation.py$}

\lstinputlisting[style=numbers]{../../../test/simulation/01nestedEstimation.py}

\subsection{\lstinline$02nestedSimulation.py$}

Available at
\href{http://biogeme.epfl.ch/examples/indicators/python/02nestedSimulation.py}{\lstinline$biogeme.epfl.ch/examples/indicators/python/02nestedSimulation.py$}

\label{sec:02nestedSimulation}
\lstinputlisting[style=numbers]{../../../test/simulation/02nestedSimulation.py}

\subsection{\lstinline$03nestedElasticities.py$}
\label{sec:03nestedElasticities}

Available at \href{http://biogeme.epfl.ch/examples/indicators/python/03nestedElasticities.py}{\lstinline$biogeme.epfl.ch/examples/indicators/python/03nestedElasticities.py$}
\lstinputlisting[style=numbers]{../../../test/simulation/03nestedElasticities.py}

\subsection{\lstinline$04nestedElasticities.py$}
\label{sec:04nestedElasticities}

Available at \href{http://biogeme.epfl.ch/examples/indicators/python/04nestedElasticities.py}{\lstinline$biogeme.epfl.ch/examples/indicators/python/04nestedElasticities.py$}
\lstinputlisting[style=numbers]{../../../test/simulation/04nestedElasticities.py}

\subsection{\lstinline$05nestedElasticities.py$}
\label{sec:05nestedElasticities}

Available at \href{http://biogeme.epfl.ch/examples/indicators/python/05nestedElasticities.py}{\lstinline$biogeme.epfl.ch/examples/indicators/python/05nestedElasticities.py$}
\lstinputlisting[style=numbers]{../../../test/simulation/05nestedElasticities.py}

\subsection{\lstinline$06nestedWTP.py$}
\label{sec:06nestedWTP}

Available at \href{http://biogeme.epfl.ch/examples/indicators/python/06nestedWTP.py}{\lstinline$biogeme.epfl.ch/examples/indicators/python/06nestedWTP.py$}
\lstinputlisting[style=numbers]{../../../test/simulation/06nestedWTP.py}






\clearpage 

\bibliographystyle{dcu}
\bibliography{../dca}

\end{document}





